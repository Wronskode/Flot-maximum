\documentclass[a4paper]{article}

% Options possibles : 10pt, 11pt, 12pt (taille de la fonte)
%                     oneside, twoside (recto simple, recto-verso)
%                     draft, final (stade de développement)

\usepackage[utf8]{inputenc}   % LaTeX, comprends les accents !
\usepackage[T1]{fontenc}      % Police contenant les caractères français
\usepackage{fontspec} % Pour les accents de Tomáš Kaiser
\usepackage[french]{babel}  
\usepackage{tikz}
\usetikzlibrary{arrows,shapes,positioning}
\usepackage{titling}
\usepackage{tikz-uml}
\usepackage{pgf-umlsd}
\usepackage{tikz}
\definecolor{myblue}{RGB}{85,114,193}
\tikzset{Dotted/.style={
    line width=1pt,
    dash pattern=on 0.01\pgflinewidth off #1\pgflinewidth,line cap=round,
    shorten >=0.3em,shorten <=0.3em},
    Dotted/.default=5}
\usetikzlibrary{arrows.meta,calc,positioning,shapes,arrows}
%\usepackage{pdflscape}
\usepackage{geometry}
\usepackage{biblatex}
\usepackage{graphicx} % Required for inserting images
\usepackage[colorlinks=true ,urlcolor=blue,urlbordercolor={0 1 1}]{hyperref}
\usepackage{float}
\usepackage{amssymb}
\usepackage{amsthm} 
\usepackage{datetime}
\usepackage{numprint}
%\usepackage{fullpage}
\addbibresource{sample.bib}
%\usepackage{minitoc}
\makeatletter
\makeatother
\newdate{frontpagedate}{30}{01}{2025} 
\begin{document}
\pagenumbering{gobble} 
\begin{center}
\vspace{2cm}
%\textsc{ Oregon State University}\\[1.5cm]
\includegraphics[width=0.4\textwidth]{UM1.jpg}~\\[1cm]
\vspace{2cm}

% Title
\hrule
\vspace{.5cm}
{\huge\bfseries{Résolution du problème du flot maximum\par}} % title of the report
\vspace{.5cm}

\hrule
\vspace{1.5cm}

\textsc{\textbf{Auteurs}}\\
\vspace{.5cm}
\centering

% add your name here
Cédric Audie\\
Marie Dalenc\\
Julien Lahoz\\
Adrien Martinelli


\vspace{1cm}

\textsc{\textbf{Encadrant}}\\
\vspace{.5cm}
\centering

% add your name here
Rodolphe Giroudeau

\vspace{4cm}

\centering \displaydate{frontpagedate} % Dags dato
\end{center}
\newpage
{\hypersetup{hidelinks}
\tableofcontents
}
\pagenumbering{arabic}  
\newpage

\section{Définitions générales}
\subsection{Réseau résiduel}

Soit $G$ un graphe avec $s$, $p$ et $c$ respectivement la source, le puits et la fonction de capacité associés au graphe $G$. Notons $N = (G,s,p,c)$ un réseau de transport dans $G$ avec un flot $f$.\\
Le réseau résiduel de $N$ et d'un flot $f$, noté $N_f$, est construit de la manière suivante :\\
Pour chaque arc $xy$ de $G$ :
\begin{itemize}
	\item
    Si $f(xy)<c(xy)$, on crée un arc $xy$ dans $G'$ avec la quantité restante disponible de la capacité: $c'(xy) = c(xy) - f(xy)$.
    \item 
    Si $f(xy)>0$ avec $x\ne s$ et $y\ne p$, on crée un arc $yx$ dans $G'$ avec la capacité qu'on peut ré-aiguiller : $c'(yx) = f(xy)$.
\end{itemize}
Ceci nous donne $N_f = (G',s,p,c')$.\\
\subsection{Chemin et flot améliorants}
\begin{itemize}
	\item
    Un chemin améliorant pour $N$ et $f$ est un chemin de $s$ à $p$ dans la réseau résiduel $N_f$.
    \item
    Soit $C = x_0,x_1,...,x_k$ un chemin améliorant dans $N_f$, le flot améliorant correspondant est :
$f'(xy) = \left\{
    \begin{array}{ll}
        \gamma & \mbox{si } xy \mbox{ est un arc de C} \\
        0 & \mbox{sinon.}
    \end{array}
\right.$
avec $\gamma = \min \{c'(x_ix_{i+1}); i \in [\![0;k-1]\!]\}$\\
\end{itemize}
\section{Différents algorithmes et exemple}
Nous prendrons l'exemple suivant pour illustrer les différents algorithmes :

\begin{center}
\begin{tikzpicture}[
	vertex/.style={circle, draw, fill=blue!20, inner sep=2pt},
	every edge/.style={draw, ->, thick}, % Ajout de flèches
	->,>=stealth % Style explicite de flèches (stealth' pour une tête nette)
	]

	% Les sommets
    \node[vertex] (S) at (0.5, 1) {S};
    \node[vertex] (A) at (2, 2) {A};
	\node[vertex] (B) at (4, 2) {B};
	\node[vertex] (C) at (2, 0) {C};
	\node[vertex] (D) at (4, 0) {D};
	\node[vertex] (P) at (5.5, 1) {P};
		
	% Les arêtes orientées
	%\draw[->, bend left=15] (A) to (B);
    \draw[->] (S) to node[midway, above left] {4} (A);
    \draw[->] (S) to node[midway, below left] {3} (C);
	\draw[->] (A) to node[midway, above] {3} (B);
	\draw[->] (A) to node[midway, left] {2} (C);
    \draw[->] (B) to node[midway, right] {3} (D);
	\draw[->] (B) to node[midway, above right] {2} (P);
	\draw[->] (C) to node[midway, below] {2} (D);
    \draw[->] (D) to node[midway, below left] {1} (A);
    \draw[->] (D) to node[midway, below right] {4} (P);
\end{tikzpicture}\end{center}


\subsection{L'algorithme de Ford-Fulkerson}

\subsubsection{Fonctionnement général}
L'algorithme de Ford-Fulkerson est le plus connu d'entre tous. Celui-ci consiste à trouver un chemin améliorant entre la source et le puits afin d'augmenter la valeur du flot. Pour cela nous aurons besoin du graphe résiduel de $G$ et $f$ qui nous permettra de savoir les capacités restantes disponibles.\\

\subsubsection{Algorithme}
Pour tout arc $xy$ du graphe $G$: on initialise la valeur du flot de $xy$ à $0$.\\
Tant qu'il existe un chemin améliorant $C$ dans le graphe résiduel de $G$ et $f$:
\begin{itemize}
	\item
    Calculer le flot améliorant $f'$ correspondant.
    \item 
    Augmenter $f$ par $f'$
\end{itemize}
Retourner $f$.\\

\subsubsection{Complexité}

\subsubsection{Avantages/inconvénients}

\subsubsection{Application sur l'exemple}
Tout d'abord, on a le chemin améliorant : $SABP$ avec une valeur de flot améliorant égal à $2$.\\
On obtient donc à droite le réseau résiduel $N_f$ :

\begin{center}
	\begin{tikzpicture}[
		vertex/.style={circle, draw, fill=blue!20, inner sep=2pt},
		highlight/.style={->, line width=3pt, draw=green!100},
		every edge/.style={draw, ->, thick},
		->,>=stealth
		]
	
		% --- Premier graphe (à gauche) ---
		\node[vertex] (S) at (0.5, 1) {S};
		\node[vertex] (A) at (2, 2) {A};
		\node[vertex] (B) at (4, 2) {B};
		\node[vertex] (C) at (2, 0) {C};
		\node[vertex] (D) at (4, 0) {D};
		\node[vertex] (P) at (5.5, 1) {P};

		\draw[highlight] (S) -- (A);
		\draw[highlight] (A) -- (B);
		\draw[highlight] (B) -- (P);
	
		\draw[->] (S) to node[midway, above left] {2/4} (A);
		\draw[->] (S) to node[midway, below left] {3} (C);
		\draw[->] (A) to node[midway, above] {2/3} (B);
		\draw[->] (A) to node[midway, left] {2} (C);
		\draw[->] (B) to node[midway, right] {3} (D);
		\draw[->] (B) to node[midway, above right] {2/2} (P);
		\draw[->] (C) to node[midway, below] {2} (D);
		\draw[->] (D) to node[midway, below left] {1} (A);
		\draw[->] (D) to node[midway, below right] {4} (P);
	
		% --- Deuxième graphe (à droite), décalé de +7 en x ---
		\node[vertex] (S2) at ($(S)+(6,0)$) {S};
		\node[vertex] (A2) at ($(A)+(6,0)$) {A};
		\node[vertex] (B2) at ($(B)+(6,0)$) {B};
		\node[vertex] (C2) at ($(C)+(6,0)$) {C};
		\node[vertex] (D2) at ($(D)+(6,0)$) {D};
		\node[vertex] (P2) at ($(P)+(6,0)$) {P};
	
		\draw[->] (S2) to node[midway, above left] {2} (A2);
		\draw[->] (S2) to node[midway, below left] {3} (C2);
		\draw[->, bend left=15] (A2) to node[midway, above] {1} (B2);
		\draw[->] (A2) to node[midway, left] {2} (C2);
		\draw[->, bend left=15] (B2) to node[midway, below] {2} (A2);
		\draw[->] (B2) to node[midway, right] {3} (D2);
		\draw[->] (C2) to node[midway, below] {2} (D2);
		\draw[->] (D2) to node[midway, below left] {1} (A2);
		\draw[->] (D2) to node[midway, below right] {4} (P2);
	
	\end{tikzpicture}
	\end{center}
	


On voit qu'il y a le chemin améliorant $SACDP$ avec un flot améliorant égal à $2$.\\
On obtient à droite le nouveau réseau résiduel $N_f$ :

\begin{center}
	\begin{tikzpicture}[
		vertex/.style={circle, draw, fill=blue!20, inner sep=2pt},
		highlight/.style={->, line width=3pt, draw=green!100},
		every edge/.style={draw, ->, thick},
		->, >=stealth
		]
	
		% --- Premier graphe (à gauche) ---
		\node[vertex] (S) at (0.5, 1) {S};
		\node[vertex] (A) at (2, 2) {A};
		\node[vertex] (B) at (4, 2) {B};
		\node[vertex] (C) at (2, 0) {C};
		\node[vertex] (D) at (4, 0) {D};
		\node[vertex] (P) at (5.5, 1) {P};

		\draw[highlight] (S) -- (A);
		\draw[highlight] (A) -- (C);
		\draw[highlight] (C) -- (D);
		\draw[highlight] (D) -- (P);
	
		\draw[->] (S) to node[midway, above left] {4/4} (A);
		\draw[->] (S) to node[midway, below left] {3} (C);
		\draw[->] (A) to node[midway, above] {2/3} (B);
		\draw[->] (A) to node[midway, left] {2/2} (C);
		\draw[->] (B) to node[midway, right] {3} (D);
		\draw[->] (B) to node[midway, above right] {2/2} (P);
		\draw[->] (C) to node[midway, below] {2/2} (D);
		\draw[->] (D) to node[midway, below left] {1} (A);
		\draw[->] (D) to node[midway, below right] {2/4} (P);
	
		% --- Deuxième graphe (à droite), décalé de +7 en x ---
		\node[vertex] (S2) at ($(S)+(6,0)$) {S};
		\node[vertex] (A2) at ($(A)+(6,0)$) {A};
		\node[vertex] (B2) at ($(B)+(6,0)$) {B};
		\node[vertex] (C2) at ($(C)+(6,0)$) {C};
		\node[vertex] (D2) at ($(D)+(6,0)$) {D};
		\node[vertex] (P2) at ($(P)+(6,0)$) {P};
	
		\draw[->] (S2) to node[midway, below left] {3} (C2);
		\draw[->, bend left=15] (A2) to node[midway, above] {1} (B2);
		\draw[->, bend left=15] (B2) to node[midway, below] {2} (A2);
		\draw[->] (B2) to node[midway, right] {3} (D2);
		\draw[->] (C2) to node[midway, right] {2} (A2);
		\draw[->] (D2) to node[midway, below left] {1} (A2);
		\draw[->] (D2) to node[midway, above] {2} (C2);
		\draw[->] (D2) to node[midway, below right] {2} (P2);
	
	\end{tikzpicture}
\end{center}

On voit qu'il y a le chemin améliorant $SCABDP$ avec un flot améliorant égal à $1$.\\
On obtient à droite le nouveau réseau résiduel $N_f$ :

\begin{center}
	\begin{tikzpicture}[
		vertex/.style={circle, draw, fill=blue!20, inner sep=2pt},
		highlight/.style={->, line width=3pt, draw=green!100},
		every edge/.style={draw, ->, thick},
		->,>=stealth
		]
	
		% --- Premier graphe (à gauche) ---
		\node[vertex] (S) at (0.5, 1) {S};
		\node[vertex] (A) at (2, 2) {A};
		\node[vertex] (B) at (4, 2) {B};
		\node[vertex] (C) at (2, 0) {C};
		\node[vertex] (D) at (4, 0) {D};
		\node[vertex] (P) at (5.5, 1) {P};

		\draw[highlight] (S) -- (C);
		\draw[highlight] (C) -- (A);
		\draw[highlight] (A) -- (B);
		\draw[highlight] (B) -- (D);
		\draw[highlight] (D) -- (P);
	
		\draw[->] (S) to node[midway, above left] {4/4} (A);
		\draw[->] (S) to node[midway, below left] {1/3} (C);
		\draw[->] (A) to node[midway, above] {3/3} (B);
		\draw[->] (A) to node[midway, left] {1/2} (C);
		\draw[->] (B) to node[midway, right] {1/3} (D);
		\draw[->] (B) to node[midway, above right] {2/2} (P);
		\draw[->] (C) to node[midway, below] {2/2} (D);
		\draw[->] (D) to node[midway, below left] {1} (A);
		\draw[->] (D) to node[midway, below right] {3/4} (P);
	
		% --- Deuxième graphe (à droite), décalé de +7 en x ---
		\node[vertex] (S2) at ($(S)+(6,0)$) {S};
		\node[vertex] (A2) at ($(A)+(6,0)$) {A};
		\node[vertex] (B2) at ($(B)+(6,0)$) {B};
		\node[vertex] (C2) at ($(C)+(6,0)$) {C};
		\node[vertex] (D2) at ($(D)+(6,0)$) {D};
		\node[vertex] (P2) at ($(P)+(6,0)$) {P};
	
		\draw[->] (S2) to node[midway, below left] {2} (C2);
		\draw[->, bend right=15] (A2) to node[midway, left] {1} (C2);
		\draw[->] (B2) to node[midway, below] {3} (A2);
		\draw[->, bend left=15] (B2) to node[midway, right] {2} (D2);
		\draw[->, bend right=15] (C2) to node[midway, right] {1} (A2);
		\draw[->] (D2) to node[midway, below left] {1} (A2);
		\draw[->, bend left=15] (D2) to node[midway, left] {1} (B2);
		\draw[->] (D2) to node[midway, above] {2} (C2);
		\draw[->] (D2) to node[midway, below right] {1} (P2);
	
	\end{tikzpicture}
	\end{center}
	

Dans ce dernier réseau résiduel, si nous partons de la source $S$ nous ne pouvons atteindre que les sommets $A$ et $C$. Comme nous ne pouvons plus atteindre le puits $P$, cela signifie que le flot obtenu est maximal.\\
Finalement la coupe minimale de ce réseau est l'ensemble $\{S,A,C\}$ et son flot maximal est de valeur $5$.\\

\subsection{L'algorithme d'Edmonds-Karp}

\subsubsection{Fonctionnement général}
L'algorithme d'Edmonds-Karp est une spécificité de celui de Ford-Fulkerson. Celui-ci consiste à trouver le plus court chemin améliorant entre la source et le puits afin d'augmenter la valeur du flot. Pour le trouver, il suffira d'utiliser un parcours en largeur.\\

\subsubsection{Algorithme}
Pour tout arc $xy$ du graphe $G$: on initialise la valeur du flot de $xy$ à $0$.\\
Tant qu'il existe un plus court chemin améliorant $C$ dans le graphe résiduel de $G$ et $f$:
\begin{itemize}
	\item
    Calculer le flot améliorant $f'$ correspondant.
    \item 
    Augmenter $f$ par $f'$
\end{itemize}
Retourner $f$.\\

\subsubsection{Complexité}

\subsubsection{Avantages/inconvénients}

\subsubsection{Application sur l'exemple}
Tout d'abord, on a le chemin améliorant : $SABP$ de taille $4$ avec une valeur de flot améliorant égal à $2$.\\
On obtient donc à droite le réseau résiduel $N_f$ :

\begin{center}
	\begin{tikzpicture}[
		vertex/.style={circle, draw, fill=blue!20, inner sep=2pt},
		highlight/.style={->, line width=3pt, draw=green!100},
		every edge/.style={->, thick},
		->,>=stealth
		]
	
		% --- Premier graphe (à gauche) ---
		\node[vertex] (S) at (0.5, 1) {S};
		\node[vertex] (A) at (2, 2) {A};
		\node[vertex] (B) at (4, 2) {B};
		\node[vertex] (C) at (2, 0) {C};
		\node[vertex] (D) at (4, 0) {D};
		\node[vertex] (P) at (5.5, 1) {P};
	
		% Arcs surlignés (avec /)
		\draw[highlight] (S) -- (A);
		\draw[highlight] (A) -- (B);
		\draw[highlight] (B) -- (P);
	
		% Arcs normaux
		\draw[->] (S) to node[midway, above left] {2/4} (A);
		\draw[->] (S) to node[midway, below left] {3} (C);
		\draw[->] (A) to node[midway, above] {2/3} (B);
		\draw[->] (A) to node[midway, left] {2} (C);
		\draw[->] (B) to node[midway, right] {3} (D);
		\draw[->] (B) to node[midway, above right] {2/2} (P);
		\draw[->] (C) to node[midway, below] {2} (D);
		\draw[->] (D) to node[midway, below left] {1} (A);
		\draw[->] (D) to node[midway, below right] {4} (P);
	
		% --- Deuxième graphe (à droite), décalé ---
		\node[vertex] (S2) at ($(S)+(6,0)$) {S};
		\node[vertex] (A2) at ($(A)+(6,0)$) {A};
		\node[vertex] (B2) at ($(B)+(6,0)$) {B};
		\node[vertex] (C2) at ($(C)+(6,0)$) {C};
		\node[vertex] (D2) at ($(D)+(6,0)$) {D};
		\node[vertex] (P2) at ($(P)+(6,0)$) {P};
	
		% Aucun surlignage nécessaire ici
		\draw[->] (S2) to node[midway, above left] {2} (A2);
		\draw[->] (S2) to node[midway, below left] {3} (C2);
		\draw[->, bend left=15] (A2) to node[midway, above] {1} (B2);
		\draw[->] (A2) to node[midway, left] {2} (C2);
		\draw[->, bend left=15] (B2) to node[midway, below] {2} (A2);
		\draw[->] (B2) to node[midway, right] {3} (D2);
		\draw[->] (C2) to node[midway, below] {2} (D2);
		\draw[->] (D2) to node[midway, below left] {1} (A2);
		\draw[->] (D2) to node[midway, below right] {4} (P2);
	
	\end{tikzpicture}
	\end{center}
	


On voit qu'il y a le chemin améliorant $SACDP$ de taille $5$ avec un flot améliorant égal à $2$. Regardons si on trouve un chemin améliorant plus petit, le chemin $SCDP$ est un autre chemin améliorant de taille $4$ avec un flot améliorant égal à $2$. C'est ce chemin que nous choisissons.\\
On obtient à droite le nouveau réseau résiduel $N_f$ :


\begin{center}
	\begin{tikzpicture}[
		vertex/.style={circle, draw, fill=blue!20, inner sep=2pt},
		highlight/.style={->, line width=3pt, draw=green!100},
		every edge/.style={->, thick},
		->,>=stealth
		]
	
		% --- Premier graphe (à gauche) ---
		\node[vertex] (S) at (0.5, 1) {S};
		\node[vertex] (A) at (2, 2) {A};
		\node[vertex] (B) at (4, 2) {B};
		\node[vertex] (C) at (2, 0) {C};
		\node[vertex] (D) at (4, 0) {D};
		\node[vertex] (P) at (5.5, 1) {P};
	
		% Arcs surlignés
		\draw[highlight] (S) -- (C);
		\draw[highlight] (C) -- (D);
		\draw[highlight] (D) -- (P);
	
		% Arcs normaux
		\draw[->] (S) to node[midway, above left] {2/4} (A);
		\draw[->] (S) to node[midway, below left] {2/3} (C);
		\draw[->] (A) to node[midway, above] {2/3} (B);
		\draw[->] (A) to node[midway, left] {2} (C);
		\draw[->] (B) to node[midway, right] {3} (D);
		\draw[->] (B) to node[midway, above right] {2/2} (P);
		\draw[->] (C) to node[midway, below] {2/2} (D);
		\draw[->] (D) to node[midway, below left] {1} (A);
		\draw[->] (D) to node[midway, below right] {2/4} (P);
	
		% --- Deuxième graphe (à droite), décalé de +7 ---
		\node[vertex] (S2) at ($(S)+(6,0)$) {S};
		\node[vertex] (A2) at ($(A)+(6,0)$) {A};
		\node[vertex] (B2) at ($(B)+(6,0)$) {B};
		\node[vertex] (C2) at ($(C)+(6,0)$) {C};
		\node[vertex] (D2) at ($(D)+(6,0)$) {D};
		\node[vertex] (P2) at ($(P)+(6,0)$) {P};
	
		% Arcs normaux
		\draw[->] (S2) to node[midway, above left] {2} (A2);
		\draw[->] (S2) to node[midway, below left] {1} (C2);
		\draw[->, bend left=15] (A2) to node[midway, above] {1} (B2);
		\draw[->] (A2) to node[midway, left] {2} (C2);
		\draw[->, bend left=15] (B2) to node[midway, below] {2} (A2);
		\draw[->] (B2) to node[midway, right] {3} (D2);
		\draw[->] (D2) to node[midway, below left] {1} (A2);
		\draw[->] (D2) to node[midway, above] {2} (C2);
		\draw[->] (D2) to node[midway, below right] {2} (P2);
	
	\end{tikzpicture}
	\end{center}
	


Il n'y a plus de chemin améliorant de taille $4$ allant de $S$ à $P$. On va choisir un chemin améliorant de taille $5$ : $SABDP$ avec un flot améliorant égal à $1$.\\
Nous obtenons à droite le nouveau réseau résiduel $N_f$ :


\begin{center}
	\begin{tikzpicture}[
		vertex/.style={circle, draw, fill=blue!20, inner sep=2pt},
		highlight/.style={->, line width=3pt, draw=green!100},
		every edge/.style={->, thick},
		->,>=stealth
		]
	
		% --- Premier graphe (à gauche) ---
		\node[vertex] (S) at (0.5, 1) {S};
		\node[vertex] (A) at (2, 2) {A};
		\node[vertex] (B) at (4, 2) {B};
		\node[vertex] (C) at (2, 0) {C};
		\node[vertex] (D) at (4, 0) {D};
		\node[vertex] (P) at (5.5, 1) {P};
	
		% Arcs surlignés
		\draw[highlight] (S) -- (A);
		\draw[highlight] (A) -- (B);
		\draw[highlight] (B) -- (D);
		\draw[highlight] (D) -- (P);
	
		% Arcs normaux
		\draw[->] (S) to node[midway, above left] {3/4} (A);
		\draw[->] (S) to node[midway, below left] {2/3} (C);
		\draw[->] (A) to node[midway, above] {3/3} (B);
		\draw[->] (A) to node[midway, left] {2} (C);
		\draw[->] (B) to node[midway, right] {1/3} (D);
		\draw[->] (B) to node[midway, above right] {2/2} (P);
		\draw[->] (C) to node[midway, below] {2/2} (D);
		\draw[->] (D) to node[midway, below left] {1} (A);
		\draw[->] (D) to node[midway, below right] {3/4} (P);
	
		% --- Deuxième graphe (à droite), décalé de +7 ---
		\node[vertex] (S2) at ($(S)+(6,0)$) {S};
		\node[vertex] (A2) at ($(A)+(6,0)$) {A};
		\node[vertex] (B2) at ($(B)+(6,0)$) {B};
		\node[vertex] (C2) at ($(C)+(6,0)$) {C};
		\node[vertex] (D2) at ($(D)+(6,0)$) {D};
		\node[vertex] (P2) at ($(P)+(6,0)$) {P};
	
		% Arcs normaux
		\draw[->] (S2) to node[midway, above left] {1} (A2);
		\draw[->] (S2) to node[midway, below left] {1} (C2);
		\draw[->] (A2) to node[midway, left] {2} (C2);
		\draw[->] (B2) to node[midway, below] {3} (A2);
		\draw[->, bend left=15] (B2) to node[midway, right] {2} (D2);
		\draw[->] (D2) to node[midway, below left] {1} (A2);
		\draw[->, bend left=15] (D2) to node[midway, left] {1} (B2);
		\draw[->] (D2) to node[midway, above] {2} (C2);
		\draw[->] (D2) to node[midway, below right] {1} (P2);
	
	\end{tikzpicture}
	\end{center}
	


Dans ce dernier réseau résiduel, si nous partons de la source $S$ nous ne pouvons atteindre que les sommets $A$ et $C$. Comme nous ne pouvons plus atteindre le puits $P$, cela signifie que le flot obtenu est maximal.\\
Finalement la coupe minimale de ce réseau est l'ensemble $\{S,A,C\}$ et son flot maximal est de valeur $5$.\\

\subsection{L'algorithme de Dinic}

\subsubsection{Fonctionnement général}
L'algorithme de Dinic est semblable à celui d'Edmonds-Karp. Comme lui, il utilise des plus courts chemins améliorants entre la source et le puits afin d'augmenter la valeur du flot. \\Pour les trouver, il faudra renommer les sommets en fonction de leur distance par rapport à la source et garder les arcs qui relient un sommet à un sommet de distance immédiatement supérieure. Ceci nous donnera le réseau de niveau $N_L$ obtenu à partir du réseau résiduel $N_f$.\\
Nous aurons également besoin du flot bloquant. Il est défini comme suit : un flot est bloquant si $\forall C$ chemin entre la source et le puits, $\exists xy$ un arc dans $C$ où $f(xy)=c(xy)$.\\

\subsubsection{Algorithme}
Pour tout arc $xy$ du graphe $G$: on initialise la valeur du flot de $xy$ à $0$.\\
Tant qu'il existe un plus court chemin améliorant $C$ dans le graphe résiduel de $G$ et $f$:
\begin{itemize}
    \item 
	On détermine le réseau de niveau $N_L$ de $N_f$.
    \item
    Calculer le flot bloquant $f'$ correspondant.
    \item 
    Augmenter $f$ par $f'$
\end{itemize}
Retourner $f$.\\

\subsubsection{Complexité}

\subsubsection{Avantages/inconvénients}

\subsubsection{Application sur l'exemple}
Au début le flot est nul, le réseau résiduel $N_f$ à droite est donc le même que le réseau de transport $N$.\\
Nous avons aussi en dessous, le réseau de niveau $N_L$ :


% --- Deux premiers graphes côte à côte ---
\begin{center}
	\begin{tikzpicture}[
		vertex/.style={circle, draw, fill=blue!20, minimum size=15pt, inner sep=1pt},
		every edge/.style={->, thick},
		->,>=stealth
		]
	
		\def\xshift{6}
	
		% --- Graphe 1 ---
		\node[vertex] (S1) at (0.5, 1) {S};
		\node[vertex] (A1) at (2, 2) {A};
		\node[vertex] (B1) at (4, 2) {B};
		\node[vertex] (C1) at (2, 0) {C};
		\node[vertex] (D1) at (4, 0) {D};
		\node[vertex] (P1) at (5.5, 1) {P};
	
		\draw (S1) to node[midway, above left] {4} (A1);
		\draw (S1) to node[midway, below left] {3} (C1);
		\draw (A1) to node[midway, above] {3} (B1);
		\draw (A1) to node[midway, left] {2} (C1);
		\draw (B1) to node[midway, right] {3} (D1);
		\draw (B1) to node[midway, above right] {2} (P1);
		\draw (C1) to node[midway, below] {2} (D1);
		\draw (D1) to node[midway, below left] {1} (A1);
		\draw (D1) to node[midway, below right] {4} (P1);
	
		% --- Graphe 2 ---
		\node[vertex] (S2) at ($(S1)+(\xshift,0)$) {S};
		\node[vertex] (A2) at ($(A1)+(\xshift,0)$) {A};
		\node[vertex] (B2) at ($(B1)+(\xshift,0)$) {B};
		\node[vertex] (C2) at ($(C1)+(\xshift,0)$) {C};
		\node[vertex] (D2) at ($(D1)+(\xshift,0)$) {D};
		\node[vertex] (P2) at ($(P1)+(\xshift,0)$) {P};
	
		\draw (S2) to node[midway, above left] {4} (A2);
		\draw (S2) to node[midway, below left] {3} (C2);
		\draw (A2) to node[midway, above] {3} (B2);
		\draw (A2) to node[midway, left] {2} (C2);
		\draw (B2) to node[midway, right] {3} (D2);
		\draw (B2) to node[midway, above right] {2} (P2);
		\draw (C2) to node[midway, below] {2} (D2);
		\draw (D2) to node[midway, below left] {1} (A2);
		\draw (D2) to node[midway, below right] {4} (P2);
	
	\end{tikzpicture}
	\end{center}
	\begin{center}
	\begin{tikzpicture}[
		vertex/.style={circle, draw, fill=blue!20, minimum size=15pt, inner sep=1pt},
		highlight/.style={->, line width=3pt, draw=yellow},
		every edge/.style={->, thick},
		->,>=stealth
		]
	
		\node[vertex] (S3) at (0.5, 1) {S:0};
		\node[vertex] (A3) at (2, 2) {A:1};
		\node[vertex] (B3) at (4, 2) {B:2};
		\node[vertex] (C3) at (2, 0) {C:1};
		\node[vertex] (D3) at (4, 0) {D:2};
		\node[vertex] (P3) at (5.5, 1) {P:3};
	
		\draw (S3) to node[midway, above left] {4} (A3);
		\draw (S3) to node[midway, below left] {3} (C3);
		\draw (A3) to node[midway, above] {3} (B3);
		\draw (B3) to node[midway, right] {2} (P3);
		\draw (C3) to node[midway, below] {2} (D3);
		\draw (D3) to node[midway, below right] {4} (P3);
	
	\end{tikzpicture}
	\end{center}
	



On a donc un chemin améliorant $SABP$ de valeur $2$ ainsi qu'un chemin améliorant $SCDP$ de valeur $2$.

\begin{center}
	\begin{tikzpicture}[
		vertex/.style={circle, draw, fill=blue!20, minimum size=15pt, inner sep=1pt},
		every edge/.style={->, thick},
		->,>=stealth
		]
	
		% Graphe 1
		\node[vertex] (S1) at (0.5, 1) {S};
		\node[vertex] (A1) at (2, 2) {A};
		\node[vertex] (B1) at (4, 2) {B};
		\node[vertex] (C1) at (2, 0) {C};
		\node[vertex] (D1) at (4, 0) {D};
		\node[vertex] (P1) at (5.5, 1) {P};
	
		\draw (S1) -- (A1) node[midway, rectangle, fill=white, above left] {2/4};
		\draw (S1) -- (C1) node[midway, rectangle, fill=white, below left] {2/3};
		\draw (A1) -- (B1) node[midway, rectangle, fill=white, above] {2/3};
		\draw (A1) -- (C1) node[midway, rectangle, fill=white, left] {2};
		\draw (B1) -- (D1) node[midway, rectangle, fill=white, right] {3};
		\draw (B1) -- (P1) node[midway, rectangle, fill=white, above right] {2/2};
		\draw (C1) -- (D1) node[midway, rectangle, fill=white, below] {2/2};
		\draw (D1) -- (A1) node[midway, rectangle, fill=white, below left] {1};
		\draw (D1) -- (P1) node[midway, rectangle, fill=white, below right] {2/4};
	
		% Graphe 2 à droite
		\begin{scope}[xshift=6cm]
			\node[vertex] (S2) at (0.5, 1) {S};
			\node[vertex] (A2) at (2, 2) {A};
			\node[vertex] (B2) at (4, 2) {B};
			\node[vertex] (C2) at (2, 0) {C};
			\node[vertex] (D2) at (4, 0) {D};
			\node[vertex] (P2) at (5.5, 1) {P};
	
			\draw (S2) -- (A2) node[midway, rectangle, fill=white, above left] {2};
			\draw (S2) -- (C2) node[midway, rectangle, fill=white, below left] {1};
			\draw[->, bend left=15] (A2) to node[midway, rectangle, fill=white, above] {1} (B2);
			\draw (A2) -- (C2) node[midway, rectangle, fill=white, left] {2};
			\draw[->, bend left=15] (B2) to node[midway, rectangle, fill=white, below] {2} (A2);
			\draw (B2) -- (D2) node[midway, rectangle, fill=white, right] {3};
			\draw (D2) -- (C2) node[midway, rectangle, fill=white, above] {2};
			\draw (D2) -- (A2) node[midway, rectangle, fill=white, below left] {1};
			\draw (D2) -- (P2) node[midway, rectangle, fill=white, below right] {2};
		\end{scope}
	\end{tikzpicture}
	\begin{tikzpicture}[
		vertex/.style={circle, draw, fill=blue!20, minimum size=15pt, inner sep=1pt},
		every edge/.style={->, thick},
		->,>=stealth
		]
	
		% Graphe 3 seul avec distances
		\node[vertex] (S3) at (0.5, 1) {S:0};
		\node[vertex] (A3) at (2, 2) {A:1};
		\node[vertex] (B3) at (4, 2) {B:2};
		\node[vertex] (C3) at (2, 0) {C:1};
		\node[vertex] (D3) at (4, 0) {D:3};
		\node[vertex] (P3) at (5.5, 1) {P:4};
	
		\draw (S3) -- (A3) node[midway, rectangle, fill=white, above left] {2};
		\draw (S3) -- (C3) node[midway, rectangle, fill=white, below left] {1};
		\draw (A3) -- (B3) node[midway, rectangle, fill=white, above] {1};
		\draw (B3) -- (D3) node[midway, rectangle, fill=white, right] {3};
		\draw (D3) -- (P3) node[midway, rectangle, fill=white, below right] {2};
	\end{tikzpicture}
	\end{center}
	
Nous avons un chemin améliorant $SABDP$ de valeur $1$.

% --- Deux premiers graphes côte à côte ---
\begin{center}
	\begin{tikzpicture}[
		vertex/.style={circle, draw, fill=blue!20, minimum size=15pt, inner sep=1pt},
		every edge/.style={->, thick},
		->,>=stealth
		]
	
		\def\xshift{6}
	
		% --- Graphe 1 ---
		\node[vertex] (S1) at (0.5, 1) {S};
		\node[vertex] (A1) at (2, 2) {A};
		\node[vertex] (B1) at (4, 2) {B};
		\node[vertex] (C1) at (2, 0) {C};
		\node[vertex] (D1) at (4, 0) {D};
		\node[vertex] (P1) at (5.5, 1) {P};
	
		\draw (S1) to node[midway, above left] {3/4} (A1);
		\draw (S1) to node[midway, below left] {2/3} (C1);
		\draw (A1) to node[midway, above] {3/3} (B1);
		\draw (A1) to node[midway, left] {2} (C1);
		\draw (B1) to node[midway, right] {1/3} (D1);
		\draw (B1) to node[midway, above right] {2/2} (P1);
		\draw (C1) to node[midway, below] {2/2} (D1);
		\draw (D1) to node[midway, below left] {1} (A1);
		\draw (D1) to node[midway, below right] {3/4} (P1);
	
		% --- Graphe 2 ---
		\node[vertex] (S2) at ($(S1)+(\xshift,0)$) {S};
		\node[vertex] (A2) at ($(A1)+(\xshift,0)$) {A};
		\node[vertex] (B2) at ($(B1)+(\xshift,0)$) {B};
		\node[vertex] (C2) at ($(C1)+(\xshift,0)$) {C};
		\node[vertex] (D2) at ($(D1)+(\xshift,0)$) {D};
		\node[vertex] (P2) at ($(P1)+(\xshift,0)$) {P};
	
		\draw (S2) to node[midway, above left] {1} (A2);
		\draw (S2) to node[midway, below left] {1} (C2);
		\draw (A2) to node[midway, left] {2} (C2);
		\draw (B2) to node[midway, below] {3} (A2);
		\draw[bend left=15] (B2) to node[midway, right] {2} (D2);
		\draw[bend left=15] (D2) to node[midway, left] {1} (B2);
		\draw (D2) to node[midway, above] {2} (C2);
		\draw (D2) to node[midway, below left] {1} (A2);
		\draw (D2) to node[midway, below right] {1} (P2);
	
	\end{tikzpicture}
	\end{center}
	
	% --- Troisième graphe centré ---
	\begin{center}
	\begin{tikzpicture}[
		vertex/.style={circle, draw, fill=blue!20, minimum size=15pt, inner sep=1pt},
		every edge/.style={->, thick},
		->,>=stealth
		]
	
		\node[vertex] (S3) at (0.5, 1) {\hspace{0.1cm} S:0 \hspace{0.1cm}};
		\node[vertex] (A3) at (2, 2) {\hspace{0.1cm} A:1 \hspace{0.1cm}};
		\node[vertex] (B3) at (4, 2) {B:$\infty$};
		\node[vertex] (C3) at (2, 0) {\hspace{0.1cm} C:1 \hspace{0.1cm}};
		\node[vertex] (D3) at (4, 0) {D:$\infty$};
		\node[vertex] (P3) at (5.5, 1) {P:$\infty$};
	
		\draw (S3) to node[midway, above left] {1} (A3);
		\draw (S3) to node[midway, below left] {1} (C3);
	
	\end{tikzpicture}
	\end{center}
	

Le puits $P$ n'est plus atteignable, l'algorithme se termine.\\
Finalement la coupe minimale de ce réseau est l'ensemble $\{S,A,C\}$ et son flot maximal est de valeur $5$.\\


%Expliquer pour chaque algo, ce qu'il fait, sa complexité, les avantages/inconvénients et faire un exemple (prendre le même pour tous les algos)
%Faire un graphique qui montre la courbe du temps de calculs de chaque algo en fonction du nombre de sommet

\section{Création des instances}

\subsection{La classe RandomFlowNetwork}

Afin de pourvoir tester et comparer nos algorithmes, nousa vons eu besoins d'instances de flots, de tailles et de formes et de topologies differentes afin de pouvoir comparer comment se comportent les different algorithmes. Comme nous n'avons pas trouvé d'instances de flots suffisement grande en ligne, nous avons décidé de créer nos propres instances. Pour cela nous avons créé une classe \texttt{RandomFlowNetwork}.\\

Cette classe permet de générer un réseau de flux orienté en plusieurs étapes :

\begin{enumerate}
    \item Elle crée un graphe vide et commence par lui ajouter l'ensemble de ses sommets, chacun identifié par un nom ou un numéro. Le nombre de sommet $n$ crée est controlé à l'avance.
    
    \item Ensuite, elle établit des arêtes orientées entre ces sommets. Ces connexions sont ajoutées de manière aléatoire, tout en respectant certaines contraintes :
    \begin{itemize}
        \item Soit on fixe à l’avance un nombre précis de connexions ;
        \item Soit on utilise une densité, c’est-à-dire une proportion du nombre maximal de connexions possibles.
    \end{itemize}
	Ici dans le cas de nos instances, cette derniere méthode est utilisée. Nous avons ainsi pu faire varier la densité de nos graphes entre 0.1 et 0.9.
    Chaque arête reçoit une capacité aléatoire, c’est-à-dire une valeur représentant sa "capacité" ou "poids".
    
    \item Une fois les arêtes ajoutées, deux sommets distincts sont choisis au hasard : un sommet source (point de départ) et un sommet puits (point d’arrivée).
    
    \item Les arêtes qui partent directement de la source et celles qui arrivent directement au puits sont récupérée et stockées à part.
    
    \item Finalement, le graphe est retourné avec :
    \begin{itemize}
        \item Le sommet source ;
        \item Le sommet puits ;
        \item Les arêtes entre les sommets.
    \end{itemize}
\end{enumerate}

En résumé, cette classe construit un graphe orienté aléatoire, puis désigne deux extrémités pour simuler un réseau de flux. Le problème est le suivant : comment construire des graphes pertinents afin de tester les algorithmes ? 

\subsection{Tests sur les graphes générés}

Les graphes étant générés aléatoirement, on peut obtenir des graphes non connexes, des sommets non reliés à la source ou au puits, etc. Tout cela fausserait les instances. En effet, si le graphe n’est pas connexe, cela pourrait fausser les résultats : un graphe de taille $1000$ pourrait se résumer à un de taille $100$, ce que l’on ne souhaite pas. On pourrait même tomber sur des graphes où nos algorithmes ne convergeraient pas, par exemple si la source et le puits ne sont pas reliés.\\

On va donc devoir créer un ensemble de tests sur nos graphes afin de s’assurer qu’ils soient bien formés et propices à être utilisés comme instances. Cela pourra prendre du temps car nous générons les graphes aléatoirement et n’en gardons qu’une petite partie, mais ce n’est pas très grave, car le jeu d’instances n’a besoin d’être créé qu’une seule fois. Il n’est pas nécessaire que le code qui permet sa création soit bien optimisé : on le lance un soir et on récupère les instances le lendemain dans le pire des cas.\\

Nous avons donc implémenté une fonction \texttt{IsConnected} qui permet de vérifier les trois critères suivants sur le graphe :
\begin{enumerate}
	\item Depuis la source, on peut atteindre n'importe quel sommet du graphe.
	\item Depuis n'importe quel sommet du graphe, on peut atteindre le puits.
	\item Le graphe est connexe.
\end{enumerate}
Cela nous permet de nous assurer que le graphe que nous avons généré est bien formé. En effet, l'algorithme convergera car le premier point nous assure qu'on peut créer un flot sur ce graphe (la source est bien connectée au puits). Les points 2 et 3 nous permettent d'être sûrs que sur un graphe de taille $n$, chaque sommet aura bien son importance et sera bien pris en compte par le calcul. Le graphe ne peut donc pas se résumer à une version plus petite de lui-même.\\



\end{document}

